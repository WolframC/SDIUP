\ifx\SUM\undefined
\documentclass[journal=jacsat,manuscript=article]{achemso}

\usepackage[version=3]{mhchem}
\usepackage{booktabs}
\usepackage{float}
\usepackage{amsmath}

\newcommand*\mycommand[1]{\texttt{\emph{#1}}}

\author{Cong Wen}
\author{Xiao Tan}
\author{Wen-Hao Li}
\author{Hong-Jin Chen}
\author{Jian-Tao Zai}
%\altaffiliation{Current address: Dormitary X13 104}
\affiliation[SJTU]
	{School of Chemistry and Chemical Engineering, Shanghai Jiao Tong University, Shanghai 200240, P. R. China}
\email{zaijiantao@sjtu.edu.cn}
\phone{+86-21-34202642}

\title{An Optimization Method for Experimental Conditions of Spectrophotometric Determination of Metal Complexes}

\abbreviations{SP,phen}
\keywords{Spectrophotometric, determination of Ions, orthogonal design, Job method, molar ratio method}

\begin{document}
\fi

\begin{abstract}

\textsf{Spectrophotometry} is an attractive method to determine the concentration and composition of metal ions, which meets the requirement of green chemistry more than most traditional method such as titration method and gravimetric method. But when apply spectrophotometry practically, there are many options in chromogenic agent, pH and chromogenic time, etc. So it's very meaningful to develope effective method to optimize the experimental condition. We choose iron as cation and phenanthroline as chromogenic agent, we apply \textsf{orthogonal experimental design} to optimize the combination of conditions. Then we used and compared \textsf{Job method} and \textsf{molar ratio method} in the determination of the composition of iron complex and copper complex, we hold the opinion that molar ratio method is more general and more effective in most cases. Under the optimized condition, we can plot a standard curve with which we can reasonably predict the concentration of metal ion, and the purity of our self-made iron(II) oxalate dihydrate. Then spectrophotometry method and titration method are compared using the solid sample, and the results are very accurate by validating each other. 

\end{abstract}

\ifx\SUM\undefined
\bibliography{reference}
\end{document}
\fi