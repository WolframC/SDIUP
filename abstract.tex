\ifx\SUM\undefined
\documentclass[journal=jacsat,manuscript=article]{achemso}

\usepackage[version=3]{mhchem}
\usepackage{booktabs}
\usepackage{float}
\usepackage{amsmath}

\newcommand*\mycommand[1]{\texttt{\emph{#1}}}

\author{Cong Wen}
\author{Xiao Tan}
\author{Wen-Hao Li}
\author{Hong-Jin Chen}
\author{Jian-Tao Zai}
%\altaffiliation{Current address: Dormitary X13 104}
\affiliation[SJTU]
	{School of Chemistry and Chemical Engineering, Shanghai Jiao Tong University, Shanghai 200240, P. R. China}
\email{zaijiantao@sjtu.edu.cn}
\phone{+86-21-34202642}

\title{An Optimization Method for Experimental Conditions of Spectrophotometric Determination of Metal Complexes}

\abbreviations{SP,phen}
\keywords{Spectrophotometric, determination of Ions, orthogonal design, Job method, molar ratio method}

\begin{document}
\fi

\begin{abstract}

\textsf{Spectrophotometry} is an attractive alternative to titration method and gravimetric method when determining the concentration of metal ions, but there are many options in chromogenic agent, pH and chromogenic time, we choose phenanthroline as chromogenic agent and apply \textsf{orthogonal experimental design} to optimize the combination. We then used and compared \textsf{Job method} with \textsf{molar ratio method} in the determination of the composition of Tris(1,10-phenantholine) iron, we hold the opinion that molar ratio method is more general and more effective in this experiment. Under the optimized condition, we can plot a standard curve from measurement of ferric standard solutions of gradiently ascending consentration, with which we can reasonably predict the purity of our self-made iron(II) oxalate dihydrate; we get results very close to what is measured using titration method, while titration method behaves much worse in respect of meeting the requirements of green chemistry.

\end{abstract}

\ifx\SUM\undefined
\bibliography{reference}
\end{document}
\fi