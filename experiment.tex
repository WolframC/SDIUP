\ifx\SUM\undefined
\documentclass[journal=jacsat,manuscript=article]{achemso}

\usepackage[version=3]{mhchem}
\usepackage{booktabs}
\usepackage{float}
\usepackage{amsmath}

\newcommand*\mycommand[1]{\texttt{\emph{#1}}}

\author{Cong Wen}
\author{Xiao Tan}
\author{Wen-Hao Li}
\author{Hong-Jin Chen}
\author{Jian-Tao Zai}
%\altaffiliation{Current address: Dormitary X13 104}
\affiliation[SJTU]
	{School of Chemistry and Chemical Engineering, Shanghai Jiao Tong University, Shanghai 200240, P. R. China}
\email{zaijiantao@sjtu.edu.cn}
\phone{+86-21-34202642}

\title{An Optimization Method for Experimental Conditions of Spectrophotometric Determination of Metal Complexes}

\abbreviations{SP,phen}
\keywords{Spectrophotometric, determination of Ions, orthogonal design, Job method, molar ratio method}

\begin{document}
\fi

\section{Experimental}
\paragraph{Synthesis of \ce{FeC2O4}$\cdot$2\ce{H2O}}
\ce{(NH4)2Fe(SO4)2} (10.00g) was dissolved in 15mL deionized water, and the solution was acidized with 2mol/L \ce{H2SO4} (3mL); the beaker was heated until \ce{(NH4)2Fe(SO4)2} dissolved completely. Then 1mol/L \ce{H2C2O4} (60mL) was added into the beaker, and the solution was heated until it boiled, stirring continuously was necessary to prevent splashing. After standing precipitation for a while, the supernatant should be poured out. Then the precipitation was well washed by deionized water and collected by suction filtration. The product was washed by acetone twice and then dried and weighed for further use.
\paragraph{Orthogonal experiment}
 We studied four factors that have five levels each, and they are shown in Table~\ref{Tab.Fac} below.
\begin{table}[H]
  \caption{Factors}
  \label{Tab.Fac}
  \begin{tabular}{lcccc}
    \toprule
    Index & $\lambda /nm$ & $V_{\ce{phen}}/mL$ & $V_{\ce{NaAc}}/mL$ & T/min\\
    \midrule
    1     & 488           & 0.5                & 1                  & 4    \\
    2     & 498           & 1                  & 3                  & 6    \\
    3     & 508           & 2                  & 5                  & 8    \\
    4     & 518           & 3                  & 7                  & 10   \\
    5     & 528           & 4                  & 9                  & 12   \\
    \bottomrule
  \end{tabular}
\end{table}

We first prepared $0.15\%$ phen solution, 1mol/L \ce{NaAc} solution, $10\%$ hydroxylamine hydrochloride solution, $100\mu g\cdot cm^{-3}$  and $20\mu g\cdot cm^{-3}$ ferric standard solution. The \ce{NaAc} solution could be used to roughly adjust pH and hydroxylamine hydrochloride solution was used to prevent iron(II) from being oxidized. According to the chosen factors, we esablished a $L_{25}(5^6)$ orthogonal table, which is shown in Table~\ref{Tab.Ort}, as well as the absorbance measured according to Table~\ref{Tab.Ort}.

\paragraph{Plot standard curve}
According to the optimum combination deduced above, iron solutions of gradiently ascending concentration were prepared by adding $0.0mL$, $1.0mL$, $2.0mL$, $3.0mL$, $4.0mL$, $5.0mL$ $20\mu g\cdot mL^{-1}$ ferric standard solution into six $50mL$ volumetric flasks, and then $1mL$ hydroxylamine hydrochloride, $7mL$ \ce{NaAc}, $2mL$ $0.15\%$phen were added in order. Adjusting the wavelength of spectrophotometer to 508 nm, absorbance of each solution was measured accurately after 12 min and shown in Table~\ref{tab.Cal}.

\paragraph{Composition determination}

\subparagraph{By molar ratio method}
First $1.00\times10^{-3} mol\cdot dm^{-3}$ ferric standard solution, $1mL$ hydroxylamine hydrochloride and $7mL$\ce{NaAc} were added into eight $50mL$ beakers in order, and then $0.00mL$, $1.00mL$, $1.50mL$, $2.00mL$, $2.50mL$, $3.00mL$, $3.50mL$, $4.00mL$, $4.50mL$ $0.15\%$phen were added to each beaker, and the solutions were accurately diluted to $50mL$ with volumetric flasks. The first solution without phen added served as reference solution. Adjusting the wavelength of spectrophotometer to 508 nm, absorbance of each solution was measured accurately after 12 min and shown in Table~\ref{tab.Mrm}.

\subparagraph{By Job method}
According to the Table~\ref{tab.Jbm}, solutions were prepared and absorbance of which were measured after 12 min. The results are also shown in Table~\ref{tab.Jbm}.
\newpage
\paragraph{Purity measurement of \ce{FeC2O4}$\cdot$2\ce{H2O}}

\subparagraph{By standard curve}
0.2g self-made \ce{FeC2O4}$\cdot$2\ce{H2O} was dissolved by $15mL$ 2mol/L \ce{H2SO4} in beaker and diluted accurately to $250mL$ using a volumetric flask, and then diluted 100 times. And absorbance of which was measured and shown in Table~\ref{tab.Pcurve}.

\subparagraph{By titration method}
1.8g-2.0g \ce{FeC2O4}$\cdot$2\ce{H2O} was dissolved by $25mL$ 2mol/L \ce{H2SO4} in beaker and heated to 40-50$^oC$, then diluted accurately to $250mL$ using a volumetric flask. $25.00mL$ of the solutions were added into conical flask. And it was titrated by standard potassium permanganate solution, with a consumption of $V_1mL$. Then we used about 0.1g zinc powder to reduct Fe(III) ion, the solution was heated until it boiled for about 10 min to ensure the iron was reducted completely. And we added a small amount of \ce{H2SO4} to dissolve residual zinc powder. Then the soluton was titrated again by standard potassium permanganate solution, with a cunsumption of $V_2mL$. The only available Xiao-Jie Zhou's results in our group are shown in Table~\ref{tab.Tit} and Table~\ref{tab.CalMn}.

\ifx\SUM\undefined
\bibliography{reference}
\end{document}
\fi