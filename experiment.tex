\ifx\SUM\undefined
\documentclass[journal=jacsat,manuscript=article]{achemso}

\usepackage[version=3]{mhchem}
\usepackage{booktabs}
\usepackage{float}
\usepackage{amsmath}

\newcommand*\mycommand[1]{\texttt{\emph{#1}}}

\author{Cong Wen}
\author{Xiao Tan}
\author{Wen-Hao Li}
\author{Hong-Jin Chen}
\author{Jian-Tao Zai}
%\altaffiliation{Current address: Dormitary X13 104}
\affiliation[SJTU]
    {School of Chemistry and Chemical Engineering, Shanghai Jiao Tong University, Shanghai 200240, P. R. China}
\email{zaijiantao@sjtu.edu.cn}
\phone{+86-21-34202642}

\title{An Optimization Method for Experimental Conditions of Spectrophotometric Determination of Metal Complexes}

\abbreviations{SP,phen}
\keywords{Spectrophotometric, determination of Ions, orthogonal design, Job method, molar ratio method}

\begin{document}
\fi

\section{Experimental}

\subsection{Narrow wavelength range}
We first prepared $12$ identical solution with phen, \ce{NaAc} and hydrochloride inside without drawing attention to the concrete concentration. Then add $1.00$cm$^{-3}$ $1.00\times10^{-3}$mol$\cdot$dm$^{-3}$ ferric standard solution and dilute to $50$mL. The absorbance were measured respectively after the same time interval, which are shown in Table~\ref{Tab.Uni}. We can see that the optimal wavelength must lie in $490$-$520$nm.

\subsection{Orthogonal experiment}
We studied four factors that have five levels each, and they are shown in Table~\ref{Tab.Fac} below.
\begin{table}[H]
	\caption{Factors}
	\label{Tab.Fac}
	\begin{tabular}{lcccc}
		\toprule
		Index & $\lambda$ /nm & \emph{V$_{phen}$}/mL & \emph{V$_{\ce{NaAc}}$}/mL & \emph{T}/min\\
		\midrule
		1     & 488           & 0.5                & 1                  & 4    \\
		2     & 498           & 1                  & 3                  & 6    \\
		3     & 508           & 2                  & 5                  & 8    \\
		4     & 518           & 3                  & 7                  & 10   \\
		5     & 528           & 4                  & 9                  & 12   \\
		\bottomrule
	\end{tabular}
\end{table}

We first prepared $0.15\%$phen solution, $1$mol/L \ce{NaAc} solution, $10\%$ hydroxylamine hydrochloride solution, $100\mu$g$\cdot$cm$^{-3}$ and $20\mu$g$\cdot$cm$^{-3}$ ferric standard solution. The \ce{NaAc} solution could be used to roughly adjust \emph{pH} and hydroxylamine hydrochloride solution was used to prevent iron(\uppercase\expandafter{\romannumeral2}) from being oxidized. According to the chosen factors, we esablished a \emph{L$_{25}(5^6)$} orthogonal table, which is shown in Table~\ref{Tab.Ort}, as well as the absorbance measured according to Table~\ref{Tab.Ort}.

\subsection{Composition determination}

\paragraph{By molar ratio method}
First $1.00\times10^{-3}$mol$\cdot$dm$^{-3}$ ferric standard solution, $1$mL hydroxylamine hydrochloride and $7$mL \ce{NaAc} were added into eight $50$mL beakers in order, and then $0.00$mL, $1.00$mL, $1.50$mL, $2.00$mL, $2.50$mL, $3.00$mL, $3.50$mL, $4.00$mL, $4.50$mL $0.15\%$phen were added to each beaker, and the solutions were accurately diluted to $50mL$ with volumetric flasks. The first solution without phen added served as reference solution. Adjusting the wavelength of spectrophotometer to $508$nm, absorbance of each solution was measured accurately after $12$min and shown in Table~\ref{Tab.Mrm}.

\paragraph{By Job method}
According to the Table~\ref{Tab.Jbm} and Table~\ref{Tab.Jbm2}, solutions were prepared and absorbance of which were measured after $12$min. The results are also shown in Table~\ref{Tab.Jbm} and Table~\ref{Tab.Jbm2}.


\subsection{Purity measurement of \ce{FeC2O4}$\cdot$2\ce{H2O}}

\paragraph{Synthesis of \ce{FeC2O4}$\cdot$2\ce{H2O}}
$10.00$g \ce{(NH4)2Fe(SO4)2} was dissolved in $15$mL deionized water, and the solution was acidized with $3$mL $2$mol/L \ce{H2SO4}; the beaker was heated until \ce{(NH4)2Fe(SO4)2} dissolved completely. Then $60$mL $1$mol/L \ce{H2C2O4} was added into the beaker, and the solution was heated until it boiled, stirring continuously was necessary to prevent splashing. After standing precipitation for a while, the supernatant should be poured out. Then the precipitation was well washed by deionized water and collected by suction filtration. The product was washed by acetone twice and then dried and weighed for further use.

\paragraph{Spectrophotometric method}

\subparagraph{Plot standard curve}
According to the optimum combination deduced above, iron solutions of gradiently ascending concentration were prepared by adding $0.0$mL, $1.0$mL, $2.0$mL, $3.0$mL, $4.0$mL, $5.0$mL $20\mu$g$\cdot$mL$^{-1}$ ferric standard solution into six $50$mL volumetric flasks, and then $1$mL hydroxylamine hydrochloride, $7$mL \ce{NaAc}, $2$mL $0.15\%$phen were added in order. Adjusting the wavelength of spectrophotometer to $508$nm, absorbance of each solution was measured accurately after $12$min and shown in Table~\ref{tab.Cal}.

\subparagraph{Concentration determination}
$0.2$g self-made \ce{FeC2O4}$\cdot$2\ce{H2O} was dissolved by $15$mL $2$mol/L \ce{H2SO4} in beaker and diluted accurately to $250$mL using a volumetric flask, and then diluted $100$ times. And absorbance of which was measured and shown in Table~\ref{tab.Pcurve}.

\paragraph{Titration method}


$1.8$g-$2.0$g \ce{FeC2O4}$\cdot$2\ce{H2O} was dissolved by $25$mL $2$mol/L \ce{H2SO4} in beaker and heated to $40$-$50^\text{o}$C, then diluted accurately to $250$mL using a volumetric flask. $25.00$mL of the solutions were added into conical flask. And it was titrated by standard potassium permanganate solution, with a consumption of \emph{V$_1$}mL. Then we used about $0.1$g zinc powder to reduct Fe(\uppercase\expandafter{\romannumeral3}) ion, the solution was heated until it boiled for about $10$min to ensure the iron was reducted completely. And we added a small amount of \ce{H2SO4} to dissolve residual zinc powder. Then the soluton was titrated again by standard potassium permanganate solution, with a cunsumption of \emph{V$_2$}mL. The only available Xiao-Jie Zhou's results in our group are shown in Table~\ref{tab.Tit} and Table~\ref{tab.CalMn}.


\ifx\SUM\undefined
\bibliography{reference}
\end{document}
\fi