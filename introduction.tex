\ifx\SUM\undefined
\documentclass[journal=jacsat,manuscript=article]{achemso}

\usepackage[version=3]{mhchem}
\usepackage{booktabs}
\usepackage{float}
\usepackage{amsmath}

\newcommand*\mycommand[1]{\texttt{\emph{#1}}}

\author{Cong Wen}
\author{Xiao Tan}
\author{Wen-Hao Li}
\author{Hong-Jin Chen}
\author{Jian-Tao Zai}
%\altaffiliation{Current address: Dormitary X13 104}
\affiliation[SJTU]
	{School of Chemistry and Chemical Engineering, Shanghai Jiao Tong University, Shanghai 200240, P. R. China}
\email{zaijiantao@sjtu.edu.cn}
\phone{+86-21-34202642}

\title{An Optimization Method for Experimental Conditions of Spectrophotometric Determination of Metal Complexes}

\abbreviations{SP,phen}
\keywords{Spectrophotometric, determination of Ions, orthogonal design, Job method, molar ratio method}

\begin{document}
\fi

\section{Introduction}

Metal is a indispensable component of this colorful world and our life and usually exists as complexes. So we always need a green, effective and efficient method to determine composition and concentration of metal complex of ordinary, micro, trace quantity in samples such as minerals, blood plasma or serum\cite{Ramsay1957The,Wong1923COLORIMETRIC} and plants\cite{Reis1994Multicommutation, Suo-YiHuang2004DeterminationofIron}. % more ref is better
As is widely accepted, titration method and gravimetry method behave very well when determining metal ions of ordinary quantity. However, they both fail to work if micro or trace analysis is required. Futhermore, neither redox titration nor coordination titration is a preferred choice for the green determination of iron; gravimetry method is complicated and finicky to operate, as well as not energy-saving.
Spectrophotometry based on Beer-Lambert Law promise a larger potential as green and effecive methods to determine metal ions of either ordinary or trace quantity\cite{King1991Spectrophotometric,Carter1971Spectrophotometric,T1975Nitrosophenol}. But practically there are so many options in chromogenic agent \cite{Yoe1944Colorimetric,Stookey1970Ferrozine,Er-KunShang20134}, and other factors to vary such as chromogenic time, properties of the determination system, etc.

As a highly representative metal, iron is the most used metal in the world, and is broadly distributed in nature and human bodies. It's cheap and often suitable for us to explore general determination method.
So, to carry out our work and illustrate our ideas conveniently, we choose iron(\uppercase\expandafter{\romannumeral2}) together with phenanthroline to demonstrate the great role of orthogonal design in optimizing experimental conditions, namely, $\lambda$(wavelength), \emph{pH}(amount of \ce{NaAc}), \emph{T}(chromogenic time) and \emph{V$_{phen}$}(amount of phenanthroline).
To show advantages and disadvantages of two method for determining the composition of metal complexes, we choose complex of copper(\uppercase\expandafter{\romannumeral2}) and sulfosalicyclic acid together with iron(\uppercase\expandafter{\romannumeral2}) complex metioned above to form a bright contrast, and reveal the relationship between coordination number and the better method.
After that, we'll determine the purity of some self-made \ce{FeC2O4}$\cdot$2\ce{H2O} solid by both titration method and spectrophotometry. While titration method produced a lot of waste, spectrophotometry tackles with the problem using a simple standard curve. And two results being in conformity will inform us that the improved spectrophotometry method behaves very well in determining concentration of ion and is undoubtedly the most direct approach to meet the requirements of green chemistry. The experiment outline is sketched below.
\begin{enumerate}
    \item Find the optimization conditions using orthogonal design
    \item Determine compositon of metal complex
    \begin{enumerate}
        \item Job method
        \item Molar ratio method
    \end{enumerate}
    \item Determine concentration of metal ion
    \begin{enumerate}
        \item Spectrophotometric method
        \item Titration method
    \end{enumerate}
\end{enumerate}


\ifx\SUM\undefined
\bibliography{reference}
\end{document}
\fi