\section{Conclusion}
 We report that using orthogonal experimental design to optimize the experimental conditions is practical. And we applied two methods to determine the composition of two complexes and knew that the ratio of phen and iron in the complex is $3:1$ and the ratio of \ce{H3SSR} and copper is $1:1$. After analyzing the curve we drew a conclusion that molar ratio method is more general and effective in determining composition of metal complexes. Then we plotted a standard curve, with which we could predict the concentration of iron in unknown sample. The result predicted by standard curve is very accurate, which is validated by measuring the purity of \ce{FeC2O4}$\cdot$2\ce{H2O} by titration. So using spectrophotometry to determine iron of micro even trace quantity is effective and meets the requirements of green chemistry. Most importantly, the chromoginice agent can be specifically chosen in various conditions so spectrophotometry can be widely applied to diverse ions of diverse quantity, as well as orthogonal design, which just requires patient work and fine choice of levels and factors.