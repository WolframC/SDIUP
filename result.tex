\section{Results and discussion}

\paragraph{Orthogonal experiment}
\begin{table}[H]
    \caption{Orthogonal Table}
    \label{Tab.Ort}
    \begin{tabular}{cccccccc}
    \toprule
    Index & $\lambda /nm$ & 2 & $V_{\ce{phen}}/mL$ & T/min & $V_{\ce{NaAc}}/mL$ & 6 & Absorbance\\
    \midrule
    01    & 488           & 1 & 0.5                & 4     & 1                  & 1 & 0.305     \\
    02    & 488           & 2 & 1                  & 6     & 3                  & 2 & 0.391     \\
    03    & 488           & 3 & 2                  & 8     & 5                  & 3 & 0.397     \\
    04    & 488           & 4 & 3                  & 10    & 7                  & 4 & 0.388     \\
    05    & 488           & 5 & 4                  & 12    & 9                  & 5 & 0.395     \\
    06    & 498           & 1 & 1                  & 8     & 7                  & 5 & 0.416     \\
    07    & 498           & 2 & 2                  & 10    & 9                  & 1 & 0.400     \\
    08    & 498           & 3 & 3                  & 12    & 1                  & 2 & 0.416     \\
    09    & 498           & 4 & 4                  & 4     & 3                  & 3 & 0.403     \\
    10    & 498           & 5 & 0.5                & 6     & 5                  & 4 & 0.319     \\
    11    & 508           & 1 & 2                  & 12    & 3                  & 4 & 0.413     \\
    12    & 508           & 2 & 3                  & 4     & 5                  & 5 & 0.433     \\
    13    & 508           & 3 & 4                  & 6     & 7                  & 1 & 0.416     \\
    14    & 508           & 4 & 0.5                & 8     & 9                  & 2 & 0.338     \\
    15    & 508           & 5 & 1                  & 10    & 1                  & 3 & 0.407     \\
    16    & 518           & 1 & 3                  & 6     & 9                  & 3 & 0.388     \\
    17    & 518           & 2 & 4                  & 8     & 1                  & 4 & 0.390     \\
    18    & 518           & 3 & 0.5                & 10    & 3                  & 5 & 0.320     \\
    19    & 518           & 4 & 1                  & 12    & 5                  & 1 & 0.392     \\
    20    & 518           & 5 & 2                  & 4     & 7                  & 2 & 0.400     \\
    21    & 528           & 1 & 4                  & 10    & 5                  & 2 & 0.305     \\
    22    & 528           & 2 & 0.5                & 12    & 7                  & 3 & 0.273     \\
    23    & 528           & 3 & 1                  & 4     & 9                  & 4 & 0.309     \\
    24    & 528           & 4 & 2                  & 6     & 1                  & 5 & 0.307     \\
    25    & 528           & 5 & 3                  & 8     & 3                  & 1 & 0.317     \\
    \bottomrule
    \end{tabular}
\end{table}

We process the data from Table~\ref{Tab.Ort} by range analysis. We use $T_i$ and $K_i$ to represent summary and average of absorbance corresponding to the $i^{th}$ level referred in Table~\ref{Tab.Fac}. The results are shown below in Table~\ref{Tab.OrtPro}.

\begin{table}[H]
    \caption{Experimental data processing}
    \label{Tab.OrtPro}
    \begin{tabular}{ccccc}
    \toprule
    & $\lambda /nm$ & $V_{\ce{phen}}/mL$ & T/min & $V_{\ce{NaAc}}/mL$\\
    \midrule
    $T_1$ & 1.876 & 1.555 & 1.850 & 1.825 \\
    $T_2$ & 1.854 & 1.915 & 1.821 & 1.844 \\
    $T_3$ & 2.007 & 1.917 & 1.858 & 1.846 \\
    $T_4$ & 1.890 & 1.942 & 1.820 & 1.893 \\
    $T_5$ & 1.511 & 1.909 & 1.889 & 1.830 \\
    $K_1$ & 0.3752 & 0.3110 & 0.3700 & 0.3650 \\
    $K_2$ & 0.3708 & 0.3830 & 0.3642 & 0.3688 \\
    $K_3$ & 0.4014 & 0.3834 & 0.3716 & 0.3692 \\
    $K_4$ & 0.3780 & 0.3884 & 0.3640 & 0.3786 \\
    $K_5$ & 0.3022 & 0.3818 & 0.3778 & 0.3660 \\
    Range & 0.0992 & 0.0774 & 0.0138 & 0.0136 \\
    \bottomrule
    \end{tabular}
\end{table}
We compare the range of each factor and draw the conclusion that wavelength and the amount of chromogenic agent have a significant impact on absorbance because their ranges are much greater, while the chromogenic time and pH have relatively weak influence. Meanwhile, Table~\ref{Tab.OrtPro} tells us the best combination of factors, on which all later absorbance measurement are based:

\begin{table}[H]
    \caption{Optimum combination of factors}
    \label{tab.Opt}
    \begin{tabular}{lcccc}
    \toprule
    & $\lambda$ & $V_{\ce{phen}}$ & T & $V_{\ce{NaAc}}$\\
    \midrule
    Optimum Index & 3(2.007) & 4(1.942) & 5(1.889) & 4(1.893)\\
    Concrete Data & 508 nm   & 3 mL     & 12 min   & 7 mL    \\
    \bottomrule
    \end{tabular}
\end{table}

\paragraph{Plot standard curve}
Using data from Table~\ref{tab.Cal}, we plot the standard curve which is shown in Figure~\ref{fig1}, the function of fittineline is $y=0.2026x+0.00076(R=0.9995)$, so it is reasonable to think it pass the origin accurately. Then we can use this curve to determine the concentration of iron in unknown sample as long as we dissolve it and measure its absorbance.
\begin{table}[H]
    \caption{standard curve data}
    \label{tab.Cal}
    \begin{tabular}{lllllll}
    \toprule
    $V_{Fe}/mL$    & 0.0 & 1.0 & 2.0 & 3.0 & 4.0 & 5.0 \\
    $c_{Fe}/\mu g\cdot mL^{-1}$
                   & 0.0 & 0.4 & 0.8 & 1.2 & 1.6 & 2.0 \\
    \midrule
    Absorbance     &0.000&0.081&0.166&0.247&0.316&0.410\\
    \bottomrule
    \end{tabular}
\end{table}

% \begin{figure}[H]
%     \includegraphics[width=\linewidth]{Fig1.pdf}
%     \caption{standard curve}
%     \label{fig1}
% \end{figure}

\paragraph{Composition determination}

\subparagraph{By molar ratio method}

 The function of fitting line on the left is $y=0.742x+0.0264(R=0.9979)$, and the function of fitting line on the right line is $y=0.254(R=1)$ which are shown below in Figure~\ref{fig2}. According to the figure, we know coordination number measured by molar ratio method is 3.07.

\begin{table}[H]
    \caption{Molar ratio method}
    \label{tab.Mrm}
    \begin{tabular}{lcccccccc}
    \toprule
    Index         &  1  &  2  &  3  &  4  &  5  &  6  &  7  &  8  \\
    \midrule
    $V_{phen}/mL$ &1.00 &1.50 &2.00 &2.50 &3.00 &3.50 &4.00 &4.50 \\
    $V_{Fe}/mL$   &1.00 &1.00 &1.00 &1.00 &1.00 &1.00 &1.00 &1.00 \\
    $\frac{V_{phen}}{V_{Fe}}$
                  &1.00 &1.50 &2.00 &2.50 &3.00 &3.50 &4.00 &4.50 \\
    Absorbance    &0.100&0.140&0.172&0.213&0.252&0.254&0.254&0.254\\
    \bottomrule
    \end{tabular}
\end{table}

% \begin{figure}[H]
%     \includegraphics[width=\linewidth]{Fig2.pdf}
%     \caption{Molar ratio method}
%     \label{fig2}
% \end{figure}

To some extent, the correlation coefficient reflects the superiority of the molar ratio method in the terms of accuracy. A reason why it is more accurate than Job method is that when $\frac{V_{phen}}{V_{Fe}}\geq3$, the slope of the line is almost 0 in theory. So compared with Job method, the molar ratio method’s correlation coefficient is closer to 1 which indicates better precision. In a word, as for studying more general coordination compound, typically whose coordination number is not 1, the molar ratio method behaves much better. According to the figure, we know coordination number measured by Job method is 2.79.

\subparagraph{By Job method}

We use $T_L$ to represent $\frac{V_{phen}}{V_{Fe}+V_{phen}}$. When $T_L<0.75$, the fitting line is $y=0.7079x-0.02057(R=0.9966)$, and when $T_L\geq3.0$, the fitting line is $y=-1.874x+1.881(R=0.9363)$ which are shown below in Figure~\ref{fig3}.

\begin{table}[H]
    \caption{Job method}
    \label{tab.Jbm}
    \begin{tabular}{lcccccccccccc}
    \toprule
    Index
    & 1 & 2 & 3 & 4 & 5 & 6 & 7 & 8 & 9 & 10& 11& 12\\
    \midrule
    $V_{phen}/mL$
    &0.0&1.8&3.0&4.0&4.8&5.3&5.7&6.0&6.2&6.4&6.6&8.0\\
    $V_{Fe}/mL$
    &8.0&6.2&5.0&4.0&3.2&2.7&2.3&2.0&1.8&1.6&1.4&0.0\\
    $T_L$
    &   &0.23&0.38&0.50&0.60&0.67&0.71&0.75&0.78&0.80&0.83& \\
    Absorbance
    &   &0.141&0.252&0.330&0.410&0.441&0.490&0.490&0.405&0.368&0.340& \\
    \bottomrule
    \end{tabular}
\end{table}

% \begin{figure}[H]
%     \includegraphics[width=\linewidth]{Fig3.pdf}
%     \caption{Job method}
%     \label{fig3}
% \end{figure}

The reason why the measured coordination number is smaller than theoretical number is when $T_L\geq3.0$, we only use four points to fit the line, and less points means less accuracy, which is unavoidable when the coordination number isn't 1. Besides, the slope of the theoretical line on the right is so great, which indicates a large relative deviation.

\paragraph{Purity measurement of \ce{FeC2O4}$\cdot$2\ce{H2O}}

\subparagraph{By standard curve}
We apply the standard curve to determine the concentration of iron in our self-made \ce{FeC2O4}$\cdot$2\ce{H2O} sample, therefore getting purity easily by simply calculating: \[\omega=\frac{c_{Fe}\times250mL\times179.9 g/mol}{m_{sample}}\].

\begin{table}[H]
    \caption{Purity Measurement by standard curve}
    \label{tab.Pcurve}
    \begin{tabular}{lcccc}
    \toprule
    Possesor      &Mass/g & Absorbance &$c_{Fe}/\mu g\cdot mL^{-1}$& purity   \\
    \midrule
    Xiao Tan      &0.2083 & 0.546      & 2.52   &$97.2\%$  \\
    Jia-Ye Luo    &0.2170 & 0.534      & 2.46   &$91.1\%$  \\
    Xin-Yang Zhao &0.2023 & 0.501      & 2.31   &$91.8\%$  \\
    Cong Wen      &0.2020 & 0.509      & 2.35   &$93.5\%$  \\
    Xiao-Jie Zhou &0.2048 & 0.517      & 2.38   &$93.4\%$  \\
    Zi-Han Zhao   &0.2184 & 0.530      & 2.44   &$89.8\%$  \\
    \bottomrule
    \end{tabular}
\end{table}

\subparagraph{By titration method}
We can first get the concentration of standard potassium permanganate solution easily using data from Table~\ref{tab.CalMn} by calculating: \[c_{\ce{KMnO4}}=\frac{2}{5}\times\frac{m_{\ce{NaC2O4}}}{134.0 g/mol\times V_{0average}}=0.01895mol/L\].

\begin{table}[H]
    \caption{standard of \ce{KMnO4}}
    \label{tab.CalMn}
    \begin{tabular}{ccc}
    \toprule
    Index\textsuperscript{\emph{a}}&$V_0/mL$&$V_{0average}/mL$\\
    \midrule
    1    & 24.44 &\\
    2    & 24.42 & 24.43\\
    3    & 24.42 &\\
    \bottomrule
    \end{tabular}\\
    \textsuperscript{\emph{a}}:Mass of sample \ce{NaC2O4}=6.2013 g
\end{table}

\begin{table}[H]
    \caption{Titration of \ce{FeC2O4}$\cdot$2\ce{H2O}}
    \label{tab.Tit}
    \begin{tabular}{ccccccc}
    \toprule
    Index\textsuperscript{\emph{a}}&$V_{1start}/mL$&$V_{1end}/mL$&$V_{1average}/mL$&$V_{2start}/mL$& $V_{2end}/mL$&$V_{2average}/mL$\\
    \midrule
    1    & 0.10 & 30.60 &       & 0.00 & 9.85 &     \\
    2    & 0.00 & 30.48 & 30.49 & 0.00 & 9.90 & 9.85\\
    3    & 0.00 & 30.50 &       & 0 00 & 9.80 &     \\
    \bottomrule
    \end{tabular}\\
    \textsuperscript{\emph{a}}:Mass of sample \ce{FeC2O4}$\cdot$2\ce{H2O} =1.8012 g;
\end{table}

Then we can easily calculate the purity by calculating: \[\omega=\frac{5}{2}\times\frac{V_{2average}\times c_{\ce{KMnO4}}\times179.9 g/mol}{m_{sample}}\times\frac{250mL}{25mL}=93.2\%\]

And the comparision of purity measured betweem titration method and spectrophotometry is shown in Table~\ref{tab.Res}.

\begin{table}[H]
    \caption{Purity comparision}
    \label{tab.Res}
    \begin{tabular}{ccc}
    \toprule
           & Titration & standard\\
    \midrule
    Purity & $93.2\%$  & $93.4\%$   \\
    \bottomrule
    \end{tabular}
\end{table}

We find the result of using standard curve is congrent with that of using titration method, therefore validating the accuracy of spectrophotometry.